%Q1
\renewcommand{\arraystretch}{1.5}
\newcommand\qone{
\vspace{-30} \begin{question}[
    \small{ \textbf{Modify Pumping Lemma for context-free Languages}}] \vspace{5pt} \\
 Consider modifying the statement of the pumping lemma, whereby 
\[ \begin{array}{clc}
    ``\mbox{1. for each } i \geq 0, uv^\mathbf{i}xy^\mathbf{i}z \in A'  \\
     \mbox{is replaced with} \\
   '\mbox{Q1. for each } i \geq 0 \mbox{ and for each } j \geq 0, uv^\mathbf{i}xy^\mathbf{j}z \in A'. \\
\end{array} \]
    
    \begin{enumerate}
             \item Prove this version does not hold for all context-free languages.
             \item Identify a non-trivial sub-class of context-free languages for which this modified 'lemma' is no different in effect to the actual pumping lemma.
    \end{enumerate} \hline
    \end{question}
}
    
    %qone%
    
    
\newcommand\qtwo{
\begin{question}[
\small{ \textbf{Modify the Kleene Star Operator}}] \\
 Consider changing the semantics of the Kleene Star operator on a Language $L$:
 ``For every language $L$, the language $L^*$ includes ach wors that is the concatenation of several (possibly zero) copies of a word in L." 

\[ \mathbf{i.e.} \ Suppose \  w \in L^* \ is \ a \ word, \ then \ it \ is \ of \ the \ form \  w=x...x \ for \ some \ x \in L \]

    With this new semantics, consider the star operation on the class of languages: regular expressions $L(R)$. 
    \\
1. Compare $L(R)^*$ with the classes of regular languages and context-free languages.

     \end{question}
}
     
    
     
    

%qthree%

\newcommand\qthree{ \begin{question}[\small{\textbf{Expressive Power}}]
    \vspace{5pt} \\ Consider a 'Right-Flooding' Turing Machine.  \vspace{5pt} \\
\small{\textbf{Two-Tape Turing Machine:}} \vspace{3} \small{First tape is input, and can read-only. Second tape is initialised empty, and can both read and write.} \vspace{3} \\ 

\small{ \textbf{'Right-Flooding' TM:}} \small{ Defined the same as the above two-tape machine, except each writing operation to the second tape (including the case when the written symbol is the same as the read symbol) is right-flooding} \vspace{5} \\
i.e. \vspace{1} \
\textbf{\indent{New symbols written are written into all (infinitely many) cells to the right of the head: not only into the cell under the head.}} \vspace{5} \\
\small{ By considering their expressive power, examine the relationships between 'Right-Flooding' Turing Machines and } \begin{enumerate}
             \item Single-Tape Turing Machines
             \item Two-Tape Turing Machines
             \item Pushdown Automota, NFAs and DFAs
    \end{enumerate} \hline
    \end{question}
}
   
   
    
%four%
\newcommand\qfour{
\begin{question}[
\small{\textbf{Rice}}] \vspace{5pt} \\
    \begin{enumerate}
       \item Is it decidable whether the language of a Turing Machine is decidable? \\
             \item Is it semi-decidable whether the language of a Turing Machine is decidable? \\ 
\end{enumerate} \hline \\
\end{question}
}

%four%

%five%
\newcommand\qfive{
\begin{question}[
\small{\textbf{SAT\_NOT\_TOTAL}}]
\vspace{5pt} \\
Let \texttt{\textbf{SAT\_NOT\_TOTAL}} be the problem taking as input a propositional formula $\Phi$ in CNF such that the assignment of all its variable to true is satisfying, and returns true if and if there is another satisfying assignment. 
\[ \mathbf{i.e.} \ An \ assignment \ satisfying \ \Phi \ that \ maps \ at \ least \ one \ variable \ to \ false. \]
    \begin{enumerate}
             \item Show that \texttt{\sat} is NP-complete.
    \end{enumerate}\hline
    \end{question}
}
%five%

%six%

\newcommand\qsix{
\begin{question}[
\small{\textbf{Arbitrary Polynomial-Time Reductions}}] \vspace{5pt} \\
In the lectures, we have discussed only polynomial-time reductions that are many-one. Give a formal definition of polynomial-time reductions that are arbitrary.
\[ \mathbf {i.e.} \ \textit{Arbitrary} \ \text{is to say without any other restrictions}  \]
    \begin{enumerate}
             \item Explain how a change from many-one polynomial-time reductions to arbitrary polynomial-time reductions affects the classes of NP, NP-hard, and NP-complete problems. 
    \end{enumerate}\hline
    \end{question}
}
    
%six%

%seven%

\newcommand\qseven{
\begin{question}[
\small{\textbf{First-Order Signature, Logic, and Graph Theory}}] \vspace{5pt} \\
 Let $\Sigma$ a (first-order) signature that consists of a single binary predicate Edge. Write a first-order logic with equality sentence.  \emph{i.e. Arbitrary} \ \text{(a formula without free variables)} \\ over $\sigma$ that, when evaluated over interpretations that are (connected and finite) acyclic directed graphs, evaluate to true if and only if the graph is a directed rooted binary tree.
    \begin{enumerate}
        \item Show that your sentence does not have this property when evaluated over all finite interpretations. \\
            \item Is it possible to write such a sentence for all finite interpretations? \hline \\
\end{enumerate} \\ \vspace{10} \\ 
\end{question}
}
    
%seven%


%qeight%\\


\newcommand\qeight{
\begin{question}[
\small{ \textbf{Algorithms and Atomic Sentences}}] \vspace{5pt} \\
Consider the Kleene Star operator on a Language $L$. We modify its semantics:
  \paragraph{ "For every language $L$, the language $L^*$ includes each word that is the concatenation of several (possibly zero) copies of a word in L." }

\[ \mathbf{i.e.} \ Suppose \  w \in L^* \ is \ a \ word, \ then \ it \ is \ of \ the \ form \  w=x...x \ for \ some \ x \in L \]

    With this new semantics, consider the class of languages: regular expressions operated on by the modified Kleene Star, $L_x$.  \begin{enumerate}
   

             \item Compare $L_x$ with the class of regular languages, $L$.
             \item Compare $L_x$ with the class of context-free languages, $L_{cf}$
         
    \end{enumerate}\hline
    \end{question}
}

    

