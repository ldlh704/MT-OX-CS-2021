\proof7{
Source: 2008 Computational Complexity Paper. If not probs partial or fuk ot

Need to write if want:

Question 3
(a) Order the following complexity classes with respect to inclusion ⊆ :
Ptime,Logspace, APspace, Pspace, ALogspace, APtime, NExptime
Which of the inclusions are known to be strict? (6 marks)
(b) Consider the following problem:
Evaluate
Input: Formula 휙 of propositional logic without variables,
i.e., a Boolean combination of atoms 0, 1.
Problem: Does 휙 evaluate to true?
– 1 –
Show that this problem is in Logspace. You need not describe your algorithm formally.
(10 marks)
(c) Define the CircuitValueProblem (CVP) and state its complexity by placing it precisely
in the hierarchy of complexity classes described in part (a). (4 marks)
(d) To every instance 퐶 of CVP we can naturally associate a formula in propositional logic
without variables and, conversely, every such formula naturally translates into an instance
of CVP.
Explain this correspondence and comment on why this does not conflict with the complexity results in part (b) and part (c). (5 marks)
\\
Proof:
Answer to question 3
(a) Logspace ⊆ ALogspace = Ptime ⊆ APtime = Pspace ⊆ APspace ⊆ NExptime.
Logspace is strictly contained in Pspace; Ptime is strictly contained in APspace as
this equals Exptime.
(b) To simplify notation, assume that the input formula is given by its syntax tree. To check
that the formula evaluates to 1 proceed as follows: Start at the root 푟 and set the current
vertex 푣 := 푟. If this is a truth value, return it. Otherwise, set 푣 to the left successor 푣
′
and proceed recursively. It is important here, that we re-use the same space used to store
푣. In particular, once the recursive call to the left successor returned a result 푡, we cannot
simply store 푡 and proceed with the right successor, as this would use too much space in
total.
Suppose a recursive call of the algorithm on a vertex 푣 has produced a truth value 푡. The
next step is to look up the predecessor 푣
′ of 푣. If 푣 is the left child of 푣
′ and the truth
value of 푣
′
is already determined by 푡, i.e. if 푣
′
is a ∨-node and 푡 = 1 or 푣
′
is a ∧-node and
푡 = 0, then return to 푣
′ and do not go into the right sub tree. Otherwise proceed with the
right successor of 푣
′
re-using the space used for the left successor.
If 푣 is the right child of 푣
′
then the truth value of 푣
′
is determined by 푡, i.e. 푣
′
evaluates
to 1 if, and only if, 푡 = 1. This follows from the fact that we only called the algorithm
– 5 –
recursively for the right successor of a node 푣 if its truth value has not already been
determined by the result for the left successor.
The algorithm is correct and only uses two pointers into the tree – one to store the current
node and another to search for a successor of the current node – and one additional bit 푡.
(c) A circuit is a connected directed acyclic graph with exactly one vertex of in-degree 0.
The vertices are labelled by exactly one of ∧,∨,¬, 0, 1 and vertices labelled by ∧,∨ have
2, vertices labelled by ¬ have 1, and vertices labelled by 0, 1 have no successor.
The Circuit Value Problem is to decide for a given circuit if it evaluates to 1. It is Ptimecomplete.
(d) The translation is obvious. However, circuits are acyclic, but not necessarily trees. So the
translation from circuits to formulae may blow up the instance exponentially.


\\






}